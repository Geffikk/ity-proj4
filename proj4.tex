\documentclass[a4paper, 11pt]{article}

\usepackage[czech]{babel}
\usepackage[utf8]{inputenc}
\usepackage[left=2cm, top=3cm, text={17cm, 24cm}]{geometry}
\usepackage{times}
\usepackage[unicode]{hyperref}
\hypersetup{colorlinks = true, hypertexnames = false}
\usepackage{microtype}

%================================================== UVOD ==========================================================%

\begin{document}
    \begin{titlepage}
        \begin{center}
            {\Huge \textsc{Vysoké učení technické v~Brně}} \\
            \vspace{\stretch{0.015}}
            {\huge \textsc
                {Fakulta informačních technologií}}\\
            
            \vspace{\stretch{0.4}}
            \LARGE
            {Typografie a~publikování\,--\,4.~projekt} \\
            \Huge{
            Bibliografická citácia}\\
            \vspace{\stretch{0.6}}
        \end{center}
    
        {\Large {\today \hfill Maroš Geffert (xgeffe00)}}
    \end{titlepage}

\section{Prečo sa učiť Matematiku}

\subsection{Definícia}
Matematika je definovaná ako štúdium zákonitosti štruktúry, zmeny a~priestoru. Z~formálneho hľadiska je matematika skúmanie axiomaticky definovaných formálnych štruktúr použitím logiky a~matematického označenia~\cite{kuvrina2011matematika}.

\subsection{Matematické disciplíny}
Uvedený zoznam matematických disciplín vonkoncom nemožno považovať za~úplný. Záujemcov o~naozaj hlboký pohľad do~členenia matematických disciplín a~ich poddisciplín možno odkázať na~AMS Mathematics Subject Classification~\cite{landau1987moments}.
\begin{itemize}
		\item Teória matematické logiky, teórie množin, číselné množiny
		\item Mocniny, odmnocniny, algebraické výrazy, algebraické rovnice a~nerovnice, funkcie
		\item Goniometria, elementárna geometria
		\item Kombinatorika, pravdepodobnosť, štatistika, postupnosti a~rady
		\item Základy diferenciálneho počtu a~integrálneho počtu
\end{itemize}
Viac o~týchto disciplínach sa môžte dočítať tu~\cite{keilova2016uvod},~\cite{kolavr2014elektronicka}.

\subsection{Prečo sa učiť matematiku}
Matematika učí kritickému mysleniu, tomu, že tvrdenia treba dokazovať. Matematika už dokázala nespočetne veľa javov ktoré si ľudia v minulosti nevedeli ani predstaviť. Niektoré si predstavíme v~tomto dokumente.\\
Potrobnejšie informácie o~tom, prečo sa učiť matematiku si môžete prečítať tu~\cite{dudley1966convergence}.

\subsubsection{\small{Simultánne rovnice}}
Simultánne modely rovníc sú typom štatistického modelu vo~forme súboru lineárnych simultánnych rovníc. Často sa používajú v~ekonometrii. Tieto modely môžeme riešit pomocou systemu rovnic ako je napríklad generalizovaná metóda momentov (GMM) a~odhad inštrumentálnych premenných (IV)~\cite{smith}, ~\cite{todhunter}.

\subsection{Matematické materiály na internete}
Kvalitných komplexných webových stránok, na~ktorých je spracováné viac matematických tém naraz je na~internete len málo. Hlavný prínos kvalitných  matematických materiálov dostupných na~internete spočívá v~tom, že~oproti klasickým učebním pomôckam ako su napr. tabule, poskytujú vďaka dynamickým a~interaktivním prvkom vyššiu názornosť. Viac informácií o~matematických materiáloch na~internete si môžete prečítať tu~\cite{robova2008webove}. 

\subsection{Matematické softwary}
Programy používané na~analýzu alebo výpočet numerických, symbolických alebo geometrických údajov.
\begin{itemize}
    \item \textbf{Matlab} - Programové prostredie a~skriptovací programovací jazyk pre~vedecko-technické výpočty. Detailnejšie je popísaný v~\cite{ANDERSSON20001}.
    \item \textbf{Maple} - Komerčný program určený pre~algebrické výpočty. Je založený na~malom jadre napísanom v~jazyku C, ktoré vytvára jazyk Maple. Detailnejšie je popísaný v~\cite{chvatalova2007maple}.
\end{itemize}

%%%%%%%%%%%%%%%%%%%%%%%%%%%%%%%% Citace %%%%%%%%%%%%%%%%%%%%%%%%%%%%%%%%

\newpage
\bibliographystyle{czechiso}
\renewcommand{\refname}{Literatúra}
\bibliography{proj4.bib}

\end{document}
